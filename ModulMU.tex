\documentclass[a4paper]{scrreport}
\usepackage[utf8]{inputenc}
\usepackage[naustrian]{babel}
\usepackage{color}
\usepackage{geometry}
%\geometry{a4paper,left=30mm,right=20mm,top=2cm,bottom=3cm}

% Schönere Aufzählungszeichen
\renewcommand*\labelitemii{\textasteriskcentered}


\title{Scriptum Messunsicherheit \\ Grundausbildung Modul MU}
\date{\today}
\author{Michael Matus \\ Bundesamt für Eich- und Vermessungswesen}

\begin{document}

\maketitle
\tableofcontents

\parindent0pt

\chapter{Einleitung}

Lorem ipsum dolor sit amet, consectetur adipiscing elit, sed do eiusmod tempor incididunt ut labore et dolore magna aliqua. Ut enim ad minim veniam, quis nostrud exercitation ullamco laboris nisi ut aliquip ex ea commodo consequat. Duis aute irure dolor in reprehenderit in voluptate velit esse cillum dolore eu fugiat nulla pariatur. Excepteur sint occaecat cupidatat non proident, sunt in culpa qui officia deserunt mollit anim id est laborum.

Lorem ipsum dolor sit amet, consectetur adipiscing elit, sed do eiusmod tempor incididunt ut labore et dolore magna aliqua. Ut enim ad minim veniam, quis nostrud exercitation ullamco laboris nisi ut aliquip ex ea commodo consequat. Duis aute irure dolor in reprehenderit in voluptate velit esse cillum dolore eu fugiat nulla pariatur. Excepteur sint occaecat cupidatat non proident, sunt in culpa qui officia deserunt mollit anim id est laborum.

\chapter{Praeliminarien}

Lorem ipsum dolor sit amet, consectetur adipiscing elit, sed do eiusmod tempor incididunt ut labore et dolore magna aliqua. Ut enim ad minim veniam, quis nostrud exercitation ullamco laboris nisi ut aliquip ex ea commodo consequat. Duis aute irure dolor in reprehenderit in voluptate velit esse cillum dolore eu fugiat nulla pariatur. Excepteur sint occaecat cupidatat non proident, sunt in culpa qui officia deserunt mollit anim id est laborum.

\section{Mathematische Grundlagen}

Lorem ipsum dolor sit amet, consectetur adipiscing elit, sed do eiusmod tempor incididunt ut labore et dolore magna aliqua. Ut enim ad minim veniam, quis nostrud exercitation ullamco laboris nisi ut aliquip ex ea commodo consequat. Duis aute irure dolor in reprehenderit in voluptate velit esse cillum dolore eu fugiat nulla pariatur. Excepteur sint occaecat cupidatat non proident, sunt in culpa qui officia deserunt mollit anim id est laborum.

\section{Partielle Ableitung}

Lorem ipsum dolor sit amet, consectetur adipiscing elit, sed do eiusmod tempor incididunt ut labore et dolore magna aliqua. Ut enim ad minim veniam, quis nostrud exercitation ullamco laboris nisi ut aliquip ex ea commodo consequat. Duis aute irure dolor in reprehenderit in voluptate velit esse cillum dolore eu fugiat nulla pariatur. Excepteur sint occaecat cupidatat non proident, sunt in culpa qui officia deserunt mollit anim id est laborum.

\chapter{Messunsicherheit nach GUM}

Lorem ipsum dolor sit amet, consectetur adipiscing elit, sed do eiusmod tempor incididunt ut labore et dolore magna aliqua. Ut enim ad minim veniam, quis nostrud exercitation ullamco laboris nisi ut aliquip ex ea commodo consequat. Duis aute irure dolor in reprehenderit in voluptate velit esse cillum dolore eu fugiat nulla pariatur. Excepteur sint occaecat cupidatat non proident, sunt in culpa qui officia deserunt mollit anim id est laborum.

\chapter{Relative Werte und Messunsicherheiten}

Lorem ipsum dolor sit amet, consectetur adipiscing elit, sed do eiusmod tempor incididunt ut labore et dolore magna aliqua. Ut enim ad minim veniam, quis nostrud exercitation ullamco laboris nisi ut aliquip ex ea commodo consequat. Duis aute irure dolor in reprehenderit in voluptate velit esse cillum dolore eu fugiat nulla pariatur. Excepteur sint occaecat cupidatat non proident, sunt in culpa qui officia deserunt mollit anim id est laborum.

\chapter{Fachgerechte Angabe von Messergebnissen - Die Önorm A 1026}

Lorem ipsum dolor sit amet, consectetur adipiscing elit, sed do eiusmod tempor incididunt ut labore et dolore magna aliqua. Ut enim ad minim veniam, quis nostrud exercitation ullamco laboris nisi ut aliquip ex ea commodo consequat. Duis aute irure dolor in reprehenderit in voluptate velit esse cillum dolore eu fugiat nulla pariatur. Excepteur sint occaecat cupidatat non proident, sunt in culpa qui officia deserunt mollit anim id est laborum.







Das Spezialverzeichniss ist eine Zusammenstellung von Arbeiten zu einen bestimmten Schlagwort.
Da heute nur mehr Wenige die meist in Kanzlei-Kurrent geschriebenen Einträge lesen können, wurde auf eine korrekte Transkription besonderes Augenmerk gelegt.  Für jeden einzelnen Eintrag (entsprechend einem Heft) gibt es je einen eigenen Eintrag mit folgenden Inhalt:

\begin{itemize}
\item Signatur
\item Titel (Inhalt) des Heftes
\item ggf. Titel der Beilagen 
\item Beschlagwortung (Sachgebiet, Spezialverzeichnis)
\item Zeitliche Zuordnung (Jahr)
\item Befasste Organisation (NEK, AEW, BEV)
\item Status (Heft im Archiv oder \textcolor{red}{fehlend})
\item ggf. ein kurzer Kommentar des Bearbeiters falls das Heft gesichtet wurde. Diese Bemerkungen sind in \textcolor{blue}{blauer} Schrift gehalten.
\end{itemize}

\chapter{Editorische Notiz}
Das vorliegende Dokument entstand über einen Zeitraum von etwa 2023 bis 20xx und soll im Wesentlichen die Präsentationen welche bei den Grundausbildung vorgetragen wurden, in transkribierter und kommentierter Form zur Verfügung stellen.

Alle Unterlagen sind unter \texttt{https://github.com/matusm/BevModulMU} hinterlegt.

Das Layout für den Druck sowie die Erstellung der PDF-Datei erfolgte mit \LaTeX.

\end{document}